\chapter{Introduction}

\section{Background and Motivation}
It is very challenging to predict stock prices over time because of its intrinsic complexity and ambiguity over financial markets. Stock price fluctuations are affected by countless variables, including economic upsets, supply and demand, and even psychological behaviors of mood swings, which have a hard time being captured in quantifiable and predictive methods. Mathematical models seldom seem to capture the intricate nature of the market dynamics perfectly. Recurrent Neural Networks, especially Long Short-Term Memory (LSTM) networks, have proved promising to solve this problem by capturing effective temporal patterns of stock price data. Nevertheless, these models have drawbacks. One major disadvantage is the inherent lag in prediction where the forecasted values mostly lag behind the actual movement of the stock prices. This delay decreases the predictability of the model and its applicability in industries that demand sharp timing for making a profit in trading decisions.

In turn, by unifying advanced neural network architectures and preprocessing techniques, this dissertation could address the challenges mentioned, improve the accuracy and the timeliness of stock price forecasts, and integrate them into algorithmic trading, potentially becoming a competitive advantage for the latter over simple lagging indicators.

\section{Objectives of the Study}
The main objectives of this dissertation are:
\begin{enumerate}
    \item To build up a more advanced stock price prediction model to be able to correctly forecast the next time step in real terms and by minimizing the lag between
observed and predicted values.
    \item To leverage RNNs and LSTM networks in conjunction with adaptive filtering and enhanced data preprocessing techniques to improve model performance.
    \item To integrate the predictive model into algorithmic trading strategies, replacing or complementing lagging indicators like Moving Averages (MAs) and Relative Strength Index (RSI) with a forward-looking predictive system.
    \item To maximize the profitability of trades by identifying optimal entry and exit points based on the predictive model’s outputs.
\end{enumerate}

\section{Scope of the Dissertation}
The scope of this dissertation includes the development, implementation, and evaluation of a stock price prediction framework using neural networks. The framework incorporates:
\begin{itemize}
    \item Advanced deep learning architectures, specifically RNNs and LSTM networks.
    \item Adaptive preprocessing methods to improve the quality of input data.
    \item Integration with algorithmic trading strategies to test the practical usability of the predictive model in financial markets.
\end{itemize}

The focus is on daily stock price data, with an emphasis on predicting short-term price movements (e.g., the next day’s closing price). While the framework is designed with stock markets in mind, the methodology can be extended to other financial instruments like commodities (e.g., oil prices) or indices.

\section{Structure of the Dissertation}

The dissertation is organized as follows:

\begin{itemize}
    \item \textbf{Abstract} \\
    Provides an overview of the study, highlighting the key objectives, methodologies, findings, and contributions of the research on stock return prediction using advanced machine learning and time series models.

    \item \textbf{Chapter 1: Introduction} \\
    Introduces the motivation behind stock return prediction, outlines the problem statement, objectives, scope of the study, and presents an overview of the dissertation structure.

    \item \textbf{Chapter 2: Literature Review} \\
    Explores existing approaches to stock price and return prediction, including statistical models like ARIMA and machine learning techniques such as LSTM and hybrid models. The chapter also identifies research gaps that motivate the proposed work.

    \item \textbf{Chapter 3: Frameworks and Models for Stock Price Prediction} \\
    Describes the proposed frameworks and model architectures in detail, including Hybrid LSTM-RLS, DNS architecture, ARIMA-LSTM Residual Integration, Multi-Feature LSTM, and Ensemble Learning with MetaNet. It also discusses data preprocessing, input-output design, training methodology, and implementation strategies.

    \item \textbf{Chapter 4: Results and Evaluation} \\
    Presents the experimental results of each model using metrics such as RMSE and $R^2$. Includes comparative analysis, actual vs. predicted plots, and interprets the strengths and limitations of each framework based on empirical evidence.

    \item \textbf{Chapter 5: Discussions and Challenges} \\
    Discusses the insights derived from the results, the challenges encountered such as prediction lag and volatility modeling issues—and reflects on the practical implications of the findings.

    \item \textbf{Chapter 6: Conclusion and Future Direction} \\
    Summarizes the research contributions and key outcomes of the dissertation. It also proposes future research directions, including model tuning, adaptive learning, and advanced generative modeling for improved prediction performance.

\end{itemize}