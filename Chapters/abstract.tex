\chapter*{Abstract}
\addcontentsline{toc}{chapter}{Abstract}

This study explored a variety of model architectures for stock return prediction, ranging from hybrid neural networks to ensemble learning and generative models. A persistent challenge across many models, particularly the enhanced Hybrid LSTM-RLS and Multi-Feature LSTM frameworks, was a one-time-step lag in predictions. Attempts to resolve this issue through alternative designs—including the DNS architecture, ARIMA-LSTM residual integration, and ARIMAX-LSTM-RLS—did not yield significant improvements and, in some cases, increased the RMSE.

Volatility modeling using GARCH proved ineffective, as it failed to capture variance in ARIMAX residuals. Conversely, the Stochastic Volatility model offered better tracking of dynamic return behavior but still exhibited occasional lag and misalignment with return magnitudes.

The most successful architecture was the Ensemble Learning Framework, which integrated ARIMAX, Random Forest, CatBoost, LightGBM, and XGBoost predictions using an LSTM-based MetaNet. This approach effectively eliminated the prediction lag and achieved the best balance between error minimization and pattern tracking, despite occasional directional errors.

GAN-based models, including Transformer GAN and RAGIC, performed poorly due to limited tuning and high sensitivity to hyperparameters. These findings underscore the difficulty of aligning high accuracy with temporal responsiveness in financial forecasting and highlight the need for further research in adaptive modeling, robust volatility estimation, and tailored GAN architectures for time series data.